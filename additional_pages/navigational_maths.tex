\begin{multicols}{2}

\section*{Index Error}

Index error is a measurement of hw accurate the sextant's vernier scale is.  That is, when the scope and the mirror are set so that the horizon meets perfectly, how far from 0.0° does the sexant read.

This value has to be added or subtracted from the sextant altitude ($H_s$), this is the first correction to be applied.

\section*{Dip Correction}

Dip corrects for hight of eye above the surface. This value has to be subtracted from the sextant altitude ($H_s$).  This is the second correction that has to be applied to the measured altitude.\\

\begin{center}
\begin{tabular}{|l|l|}
\hline
\texttt{Height of Eye} & \texttt{Dip Correction}\\
\hline
\texttt{0.0} &	\texttt{0.0}\\
\texttt{0.5}	& \texttt{0.0207182287}\\
\texttt{1.0} &	\texttt{0.0293}\\
\texttt{1.5}	& \texttt{0.0358850247}\\
\texttt{2.0} &	\texttt{0.0414364574}\\
\texttt{2.5}	& \texttt{0.0463273677}\\
\texttt{3.0} &	\texttt{0.0507490887}\\
\texttt{3.5} & \texttt{0.0548152807}\\
\texttt{4.0}	& \texttt{0.0586}\\
\hline
\end{tabular}
\end{center}


\section*{Refraction Correction}
The next correction is for refraction in the earths atmosphere. As usual this table is correct for 10°C and a pressure of 1010hPa. This correction has to be applied to apparent altitude ($H_a$). The exact values can be calculated by the following formula.
\[R_0=\cot \left( H_a + \frac{7.31}{H_a+4.4}\right)\]
For other than standard conditions calculate a correction factor for $R_0$ by: \[f=\frac{0.28P}{T+273}\] where $P$ is the pressure in hectopascal and $T$ is the temperature in °C.

\section*{Semidiameter}
 Semidiameter has to be added for lower limb sights and subtracted for upper limb sights. The value for semidiameter is tabulated in the daily pages.

To correct your sextant altitude $H_s$ do the following:
Calculate $H_a$ by
 \[H_a= H_s+I-dip\]
Where $I$ is the sextants index error. Then calculate the observed altitude $H_o$ by
\[H_o= H_a-R+P\pm SD\]
where $R$ is refraction, $P$ is parallax and $SD$ is the semidiameter.

Sight reduction tables can be downloaded from the US governments internet pages. Search for HO-229 or HO-249.  These values can also be calculated with two, relatively simple, formulas
\[ \sin H_c= \sin L \sin d + \cos L \cos d \cos LHA\]
and
\[\cos A = \frac{\sin d - \sin L \sin H_c}{\cos L \cos H_c}\]
where $A$ is the azimuth angle, $L$ is the latitude, $d$ is the declination and $LHA$ is the local hour angle. The azimuth ($Z_n$) is given by the following rule:
\begin{itemize}
    \item if the $LHA$ is greater than 180°, $Z_n=A$
    \item if the $LHA$ is less than 180°, $Z_n = 360° - A$
\end{itemize}
\end{multicols}
