DIP corrects for hight of eye over the surface. This value has to be subtracted from the sextant altitude ($H_s$). The  correction in degrees for hight of eye in meters is given by the following formula:
\[d=0.0293\sqrt{m}\]
This is the first correction (apart from index error) that has to be applied to the measured altitude.

The next correction is for refraction in the earths atmosphere. As usual this table is correct for 10°C and a pressure of 1010hPa. This correction has to be applied to apparent altitude ($H_a$). The exact values can be calculated by the following formula.
\[R_0=\cot \left( H_a + \frac{7.31}{H_a+4.4}\right)\]
For other than standard conditions calculate a correction factor for $R_0$ by: \[f=\frac{0.28P}{T+273}\] where $P$ is the pressure in hectopascal and $T$ is the temperature in °C.

 Semidiameter has to be added for lower limb sights and subtracted for upper limb sights. The value for semidiameter is tabulated in the daily pages.

To correct your sextant altitude $H_s$ do the following:
Calculate $H_a$ by
 \[H_a= H_s+I-dip\]
Where $I$ is the sextants index error. Then calculate the observed altitude $H_o$ by
\[H_o= H_a-R+P\pm SD\]
where $R$ is refraction, $P$ is parallax and $SD$ is the semidiameter.

Sight reduction tables can be downloaded from the US governments internet pages. Search for HO-229 or HO-249.  These values can also be calculated with two, relatively simple, formulas
\[ \sin H_c= \sin L \sin d + \cos L \cos d \cos LHA\]
and
\[\cos A = \frac{\sin d - \sin L \sin H_c}{\cos L \cos H_c}\]
where $A$ is the azimuth angle, $L$ is the latitude, $d$ is the declination and $LHA$ is the local hour angle. The azimuth ($Z_n$) is given by the following rule:
\begin{itemize}
    \item if the $LHA$ is greater than 180°, $Z_n=A$
    \item if the $LHA$ is less than 180°, $Z_n = 360° - A$
\end{itemize}
